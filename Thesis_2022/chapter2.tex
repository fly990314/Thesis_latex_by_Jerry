\chapter{背景知識}

\section{HyperText Markup Language}\label{s2.1}
\indent
標記式語言(Markup Lanaguage)是一種利用電腦看得懂的標示符號來定義出一份資料會如何呈現在畫面中,
而超文本標記語言(HyperText Markup Language, HTML)則是用於建立網站的標記式語言\cite{HTML}\cite{HTML-book},
透過網頁瀏覽器的讀取,可以將其轉換成視覺化的網站。
雖然它不算是程式語言的一種,卻是架構網站的重要根基。
程式碼\ref{l2.1}為一HTML檔案內容。

\begin{lstlisting}[caption=HTML範例檔案, label={l2.1}]
    1    <!DOCTYPE html> 
    2    <html>
    3        <head>
    4            <title> This is a title </title>
    5        </head>
    6        <body>
    7            <p> Hello World! </p>
    8        </body>
    9    </html>
\end{lstlisting}

\section{Node.js}\label{s2.2}
Node.js是一個能夠在伺服器端執行JavaScript的Open Source跨平台執行環境\cite{Node.js}\cite{Node.js-book},
讓JavaScript這種程式語言不僅僅只能在操作使用者端的程式碼。
它是讓使用者把僅能在瀏覽器執行JavaScript的限制,
改成可以透過電腦中的終端介面來直接執行JavaScript程式碼。
而Node.js是一個輕量的framework,
可以將多個函數包裝成模組提供給其他檔案來使用,
使用者可以用利用NPM\cite{NPM}下載模組或自己撰寫模組來實現自己期望的功能。

\subsection{Cherrio}\label{s2.2.1}
Cherrio是在Node.js的一個模組,實現了把jQuery\cite{Jquery}中Selector輕量化的功能,
需要先讀入HTML進行解析,並使用選擇器找出所需要的元件節點來作客製化的操作。
例如:若要對該網頁進行爬蟲,需要先讀入一整個完整的網頁HTML,
並利用元件的條件找出該元件並讀出該文字內容。

\subsection{Diff}\label{s2.2.2}
Diff是在Node.js的一個模組,本論文取用的Hiff開源程式庫\cite{Hiff}是以該模組為基礎來開發的,
利用其中的函數對傳入的兩個參數相互比較並回傳兩參數是否有差異,參數的類型可以是字元、陣列等等各種類別。

\section{DOM-based Locators}\label{s2.3}
在撰寫網頁自動化測試腳本時,需要透過定位元件並讓測試腳本對它進行多種操作,
定位元件的方法基本上可以分類成三種\cite{Web-Locators}:

\indent
1. Coordinate-based locators: 使用座標定位。

\indent
2. DOM-based locators: 使用文件物件模型(Document Object Model, DOM)\cite{DOM}定位。

\indent
3. Visual locators: 使用視覺定位,通常使用影像辨識技術。

\indent
本論文中談論的情境是以穩定性較高的DOM-based locators來操作網頁中的物件,其中DOM在這裡指的是網頁中的HTML樹狀結構,
而實現此定位的技術即為XPath。

\section{XML Path Language}\label{s2.4}
XML路徑語言(XML Path Language, XPath)是一種用來尋找XML文件\cite{XML}中一個或多個節點位置的查詢語言\cite{XPath}。
以XML中樹狀結構為基礎,
讓使用者用一個附帶多個條件的表達式,
找出是否有符合這些條件的路徑,
通常程式使用介面是由HTML構成的,
而HTML結構屬於XML結構的一種,
所以在網頁自動化測試中經常使用XPath定位找出元件位置\cite{Test-Case-Aging-By-XPath}並對該元件進行操作。

在XPath表達式中,最常見的是路徑表達式\cite{XPath-Selenium-Selectors}。
路徑表達式是從一個XML節點到另一個節點或一組節點的步驟順序。
如程式碼\ref{l2.2}表示,從HTML起始節點「ul」到屬性id為second-level的節點「li」,再從節點「li」到節點「div」,
其路徑表達式為\colorbox{lightgray}{/ul/li[@id=``second-level'']/div}

\begin{lstlisting}[caption=HTML結構範例, label={l2.2}]
    1    <ul> 
    2        <li id="second-level">
    3            <div> Text </div>
    4        </li>
    5    </ul>
\end{lstlisting}

路徑表達式通常由三個元件設定來組成,
以程式碼\ref{l2.2}的路徑表達式為例:

\begin{itemize}
\item[●] 一個 Axis。代表搜尋的方向,例如:路徑表達式中的\colorbox{lightgray}{/}
\item[●] 一個 Node test。可以選擇指定或不指定特定的 HTML 的標籤名稱,例如表達式中的\colorbox{lightgray}{li}
\item[●] 零個或多個 Predicate。描述搜尋節點的條件,藉此過濾出需要的節點,例如:\colorbox{lightgray}{/li[@id=``second-level'']}
\end{itemize}

\section{Chrome Developer Tools}\label{s2.5}
基於桌機瀏覽器市佔排行,本論文選用市佔率第一的Chrome瀏覽器來使用。
在瀏覽器的頁面中,點擊功能鍵F12即可打開Chrome Developer Tools\cite{Chrome-Developer-Tools}。
本工具最常用的四個功能模組為:
\begin{itemize}
    \item[●] 元素(Elements): 主要可以展示出該網站的當下HTML內文,以及每個元素對應的CSS屬性等等細部設定,
                            也可以編輯當下畫面的屬性參數,雖然更改的數值不會回傳儲存,但可以查看若更改會有什麼情況。
    \item[●] 控制台(Console): 屬於網頁版的命令提示字元。在開發網頁時,可以利用此頁面除錯,也可以直接執行JavaScript指令。
                            例如:通常在設計XPath時,會在此頁面輸入設計好的XPath並確認是否能定位到正確的元件,
    \item[●] 源代碼(Sources): 從該頁面可以知道該網站的HTML、JavaScrpit和CSS等等程式,也可以從此頁面設定斷點,
                            來增加Debug的效率。
    \item[●] 網路(Network): 在網站運行時,可以透過此頁面觀察該網頁和伺服器傳遞Request的情況,
                            也可以手動調整網路的速度來檢測網站的穩定性。
\end{itemize}

\begin{figure}[H]
    \centering
    \includegraphics[width=0.9\textwidth]{picture/ch2-chromeDeveloperTool.png}
    \caption{Chrome Developer Tools頁面範例}
    \label{f2.1}
\end{figure}

\section{Chrome Extension}\label{s2.6}

瀏覽器除了有本身基本的功能之外,如果使用者還有其他需求的話,可以去網路商城下載擴充套件或自己撰寫相關程式讓瀏覽器達到使用者額外的需求。
一般來說擴充元件\cite{Chrome-Extension}\cite{Chrome-Extension-Thesis}都是用諸如HTML、CSS和JavaScript的網路技術,另外有些還會透過瀏覽器的API來做溝通及變換畫面。
例如:目前市面上有很多人使用網頁版的Google翻譯來解決難以讀懂英文文章的困擾,
但每次都需要跳轉到網頁才可以進行翻譯,
Google翻譯擴充套件能讓使用者在當下的畫面把文字圈選起來,利用程式判斷出圈起來的文字並進行翻譯,
最後在該頁面的HTML增添一些元件來顯示出翻譯的結果供讀者參考。

\subsection{輸入元件}\label{s2.6.1}
Chrome瀏覽器提供能讓使用者和網頁互動的元件,有畫面可見的瀏覽器按鈕和網頁內容等等UI元素,也有畫面中的右鍵功能選單和快捷鍵等等被隱藏的區塊。

\subsection{安裝檔Manifest.json}\label{s2.6.2}
作為擴充元件的安裝檔,內部記錄了很多重要的程式路徑以及權限設定等等的相關資訊,
若需要將此擴充元件安裝在瀏覽器中使用,需要透過載入此檔案,讓瀏覽器可以讀入程式腳本以及相關參數。
程式碼\ref{l2.3}為安裝檔Manifest.json的基本架構。

\begin{lstlisting}[caption=Manifest.json安裝檔之範例架構, label={l2.3}]
    1{
    2   "manifest_version": 3,
    3   "name": "HIFF of element",
    4   "version": "0.1.0",
    5   "description": "compare HTML element's content extension.",
    6   "icons": {...
    7   },
    8
    9   "action": {...
    10   },
    11
    12  "permissions": [...
    13  ],
    14
    15  "background": {...
    16  },
    17
    18  "content_scripts": [
    19  ],
    20
    21  "devtools_page": "devtools.html"
    22}
\end{lstlisting}

\subsection{內部腳本}\label{s2.6.3}
在擴充元件的架構中,
每個腳本都有自己專門處理的區塊,
腳本與腳本之間的變數也都不相通,
若要傳遞資料只能用API來做溝通,
以下列出常用腳本所負責的功能:

\begin{itemize}
    \item[●] \textbf{Background Script: }
    一個擴充元件中,專門在背景頁面中執行的腳本,通常是處理程式中的主要邏輯。
    \item[●] \textbf{Content Script: }
    作為Inspected window和擴充元件其他腳本溝通的管道,
    可以利用此腳本操控指定網頁下的元件或建立元件的監聽器,
    進而提供與網頁內容相關的功能。
    \item[●] \textbf{Popup Script: }
    瀏覽器中右上角有個擴充元件的頁面按鈕,內部有個獨立的頁面,可以在裡面處理獨立的資訊和畫面。
    \item[●] \textbf{DevTools Script: }
    在瀏覽器的Developer Tools中,可以建立子頁面來建立自定義的功能或透過其他工具頁面來取得開發者相關訊息。
\end{itemize}

如圖\ref{f2.3}所示,一個擴充程式擁有一個background Page、多個content script和多個DevTools page的Panels等等其他的頁面,
所有的擴充程式腳本除了Content script需要先指定好要與哪個Tab Id的Tab Content script溝通外,其餘的都可以自由的互相溝通。

不同的腳本中都有獨立的Scope,可以在內部存儲變數,但它們是不能直接存取對方的資料的,
例如:Popup Script想要Background Script運算完的結果,
但兩個腳本之間無法透過互相溝通來傳遞變數資料,
只能透過Chrome中提供的通訊API,
讓Popup Script可以請求Background Script傳送資料,
且Background Script收到請求後,便回傳資料給Background Script。

\begin{figure}[H]
    \centering
    \includegraphics[width=0.8\textwidth]{picture/ch2-extension-script-relation.png}
    \caption{擴充程式腳本之關係圖}
    \label{f2.3}
\end{figure}