\chapter*{摘~~要}
\addcontentsline{toc}{chapter}{中文摘要}

%基本資訊

\noindent
論文名稱:一個提高判斷HTML中元件差異速度的瀏覽器擴充套件\\
頁數:XX頁\\
校所別:國立台北科技大學~資訊工程系碩士班\\
畢業時間:一百一十學年度第二學期\\
學位:碩士\\
研究生:廖昱翔\\
指導教授:陳碩漢教授\\
\hspace*{\fill}\\
\noindent
關鍵字:HTML、自動化驗收測試、瀏覽器擴充套件、比較HTML內容\\
\hspace*{\fill}\\
%
\indent
在自動化網頁測試中,會一直運用到Xpath表達式來定位在HTML文本中元件的位置。在一個大型的HTML文本中,時常會透過和網站元件的互動,從而來改變HTML的屬性或資料。一個測試人員在撰寫自動化腳本的時候,可能只會專注在單一元件屬性的變化,而較困難去找出其他更適合可以拿來當作限制條件的元素

本論文提出利用文本比較的方式加在瀏覽器Developer tools上做成一個擴充套件,除了可以在元件旁邊的頁面下看到當前節點和先前節點的比較,以及可以在主頁面下看到HTML文本所有的比較結果,另外加入了倒數計時功能,可以在不用點擊觸發比較的情況下,產生出結果,讓開發者可以根據結果來縮減Xpath長度以及耐用度,進而增加程式提升穩定性。