\chapter*{摘~~要}
\addcontentsline{toc}{chapter}{中文摘要}

%基本資訊

\noindent
論文名稱:以瀏覽器擴充套件提升網頁事件觸發後HTML 差異判斷之速度\\
頁數:三十七頁\\
校所別:國立台北科技大學~資訊工程系碩士班\\
畢業時間:一百一十學年度第二學期\\
學位:碩士\\
研究生:廖昱翔\\
指導教授:陳碩漢教授\\
\noindent
關鍵字:HTML、自動化驗收測試、瀏覽器擴充套件、比較HTML內容\\
\hspace*{\fill}\\
%
\indent
撰寫自動化網頁測試腳本的目的,
主要是利用程式來模擬使用者在操作網頁的情形,
其中為了要檢測網頁中的互動元件是否正常被觸發,
會利用部分變更的HTML內容,當作互動成功的判斷條件。
測試人員在撰寫自動化腳本時,因為元件的快速變化或只關注當下元件的變化...等等因素,
使得測試人員較難使用較有效率的方式找到較為穩定的元素限制條件。

本論文提出利用HTML文本比較的方式,建立一個加在瀏覽器開發者工具上的擴充套件,
可以在瀏覽器開發者工具的面板中看到文本對比後找到的所有相異的地方,
並詳細列出差異的屬性或標籤...等等內容。
為了保持互動結果的穩定性和減少不必要的比對結果,額外設計了計時和過濾的功能,
擴充套件的三大功能會使得開發者可以根據結果來縮減Xpath長度以及耐用度,進而增加程式提升穩定性。