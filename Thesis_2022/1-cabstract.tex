\chapter*{摘~~要}
\addcontentsline{toc}{chapter}{中文摘要}

%基本資訊

\noindent
論文名稱:以瀏覽器擴充套件提升網頁事件觸發後HTML 差異判斷之速度\\
頁數:XX頁\\
校所別:國立台北科技大學~資訊工程系碩士班\\
畢業時間:一百一十學年度第二學期\\
學位:碩士\\
研究生:廖昱翔\\
指導教授:陳碩漢教授\\
\noindent
關鍵字:HTML、自動化驗收測試、瀏覽器擴充套件、比較HTML內容\\
\hspace*{\fill}\\
%
\indent
撰寫自動化網頁測試腳本,
主要是用程式模擬使用者在操作網頁的情形,
其中為了要判斷網頁中的互動元件是否正常被觸發,
會利用部分變更的HTML內容,當作互動成功的判斷條件。
測試人員在撰寫自動化腳本時,元件的快速變化或只關注當下元件的變化...等等因素,
使得測試人員較難使用最有效率的方式找到最好的元素限制條件。

本論文提出利用HTML文本比較的方式,建立一個加在瀏覽器Developer Tools上的擴充套件,
可以在瀏覽器Developer Tools的面板中看到文本對比後找到的所有相異的地方,
並詳細列出差異的類別或標籤...等等內容。
另外加入計時和過濾功能,為避免在結果過多的情況利用過濾功能過濾多餘的結果且另用計時功能保持互動結果的準確性。
擴充套件的三大功能會使得開發者可以根據結果來縮減Xpath長度以及耐用度,進而增加程式提升穩定性。