\chapter*{摘~~要}
\addcontentsline{toc}{chapter}{中文摘要}

%基本資訊

\noindent
論文名稱:以瀏覽器擴充套件提升網頁事件觸發後HTML差異判斷之速度\\
頁數:四十頁\\
校所別:國立臺北科技大學~資訊工程系碩士班\\
畢業時間:一百一十學年度第二學期\\
學位:碩士\\
研究生:廖昱翔\\
指導教授:陳碩漢博士\\
\noindent
關鍵字:超文本標記語言、超文本標記語言比較、自動化驗收測試、瀏覽器擴充套件\\
\hspace*{\fill}\\
\indent
撰寫自動化網頁測試腳本的目的,
主要是利用程式模擬使用者在操作網頁的情境來檢查網頁是否正常,
其中為了要檢測網頁中的互動元件是否正常被觸發,
會利用因為互動而產生變化的HTML內容,當作互動成功的判斷條件。
測試人員在撰寫自動化腳本時,因為元件的快速變化或只關注當下元件的變化等等因素,
使得測試人員難以使用有效率的方式找到較為穩定的元素限制條件。

本論文提出設計一個加在瀏覽器開發者工具上的擴充套件,
並使用HTML比對來挑選出前後的變化,
最後利用下拉式選單讓使用者查看變化後有差異的屬性或標籤等等內容。
為了增加互動前後HTML的穩定性和減少不必要的比對結果,額外設計了計時和過濾的功能,
擴充套件的三大功能讓開發者可以根據比對結果來縮減XPath長度以及耐用度,
進而降低設計XPath表達式的時間以及提升腳本穩定性。