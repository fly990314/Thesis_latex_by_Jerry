\chapter*{誌~謝~}
\addcontentsline{toc}{chapter}{誌謝}

\indent
在此先感謝陳碩漢老師的指導,
因為本身高職和大學皆讀電機工程系,
大學畢業後轉戰資訊工程系。
相較於本科生,在資訊面向的基礎偏弱,
很多基本知識也都要詢問同學和查資料才知道,
但在碩士的兩年時光,除了學習一般上課和專案中需要的軟體,
也因為新實驗室的關係,資源和器材很多都需要購買新的,
頻繁的軟體安裝和陌生的硬體規格在剛開始摸索的時候,
須要花大量的時間去Google,
雖然那段時間很無助,但撐過後看到成果,就會有滿滿的成就感。

同時十分感謝口試委員鄭有進老師、梁郁珮老師,
對於我的論文提出他們的觀點及建議,提高論文的整體度,
也十分感謝吳俊青學長和林稟宸學長,
開始進入碩士生活後,就不斷的培養我專案開發的能力和程式除錯的技巧,
雖然有時想法轉不過來會讓學長們漸漸失去耐心,但最後也都有成功讓我了解。
也感謝專案上的學長、同學們、學弟妹們和同一實驗室的同學們在我論文研究和論文實驗測試給予我幫助,
若遇到技術問題會一同討論並提出自己的想法,
需要他們幫忙當測試人員也都是一口答應,
因為有了他們才讓我在碩士的路上順順利利,
尤其要特別感謝譚華暄學妹在我撰寫論文的過程中,不斷的幫助我校稿,讓我論文中的一字一句都更通順,
也希望她之後在寫自己的論文時,可以比我寫的還順利。

最後感謝父母從小到大的栽培,在研究過程也時常幫我準備食物,幫我減省時間和餐費,
有時即使很晚回家,也會準備熱湯或水果讓我肚子餓時不會沒有食物吃。
研究並撰寫論文雖然是一件困難的事,
但在一定的壓力下做完,不僅有技術上的增長,也能讓抗壓能力有進一步的提升,
期許自己在未來也會繼續往前努力。