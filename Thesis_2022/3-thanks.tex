\chapter*{誌~謝~}
\addcontentsline{toc}{chapter}{誌謝}

\indent %待補強
在此先感謝陳碩漢老師的指導,因為本身高職電機出身大學畢業後轉戰資工的轉系生,本身在資訊面向的基礎較弱,
但在碩士的兩年時光,除了學習一般上課和專案中需要的軟體,
也因為新實驗室的關係,資源和器材都需要購買新的,隨之而來的軟體安裝和日常操作都需要慢慢摸索才知道要如何使用會較好,
有器材可以提供給自己學習的機會也是非常難得的。

同時十分感謝口試委員鄭有進老師、梁郁珮老師,對於我的論文提出他們的觀點及建議,提高論文的整體度,
也十分感謝吳俊青學長和林稟宸學長,從碩一進入碩士生活之後就不斷的培養我如何設計腳本的相關知識和程式除錯的技術,
雖然有時想法轉不過來會讓學長們漸漸失去耐心,但最後也都有成功讓我了解。

也感謝專案上的學長、同學們、學弟妹們和同一實驗室中的同學們在我論文研究和論文實驗測試給予我幫助,
有遇到技術問題可以和他們討論,有學業的困難也很樂意協助我排除問題,需要他們幫忙當測試人員也都是一口答應,
因為有了他們才讓我在碩士的路上順順利利。

最後感謝父母從小到大的栽培,在研究過程也時常幫我準備食物讓我減省不少時間和餐費,
即使很晚回家有時也會準備熱湯或水果讓我肚子餓的時候可以拿來填飽肚子。

研究並撰寫論文雖然是一件困難的事,
但在一定的壓力下做完,不僅有技術上的增長,也讓抗壓能力有進一步的提升,
期許自己在未來也能繼續往前努力,而不是慢慢退步或停滯不前。