\chapter{緒論}
\section{研究背景與動機}
\indent
一個簡單的網站通常是由HTML、CSS和JavaScript來組成的,其中HTML是網站的主要架構,CSS可以調整網站視覺化的呈現,JavaScript可以使物件有可互動的特性。
在做網頁互動的時候,會用使HTML中裡面的元件做增加刪除修改,使的整個HTML變得稍微不同,
例如:在點擊網站中的按鈕,會跳出"你已點擊"一行字,這代表的是在你點擊後,HTML因為JavaScript程式添加了一個"你已點擊"的文字元件。
而另外Xpath是一個基於XML樹狀結構來讓使用者添加尋找的條件得以找到想獲得的元件,使用者可以利用表達式不同的寫法來定位目標物件。

\indent
在撰寫一個網頁自動化測試腳本,去和網頁中的元件互動是必須的,互動完通常我們常常需要定位一個或多個元件去等它出現或沒有出現,透過這個規律來設計出一連串的操作。
為了要定位元件,會使用Xpath表達式來找到我們需要的元件,但通常如果Xpath的條件設定的不好,常常會因為網頁結構改變需要去多次去調整Xpath。

\indent
通常我們為了要增加Xpath的穩定性,我們會尋找會因為某個使用者互動而影響元件屬性或一些固有的屬性來在Xpath裡面去限制它的條件,
而如果HTML文本較大,我們在尋找時範圍會變大,花的時間成本也相對高,因此為了要改善這方面的問題,需要一個能夠快速過濾一些不相關且沒有變化的元件,讓我們方便挑選出有產生改變的元件來當我們的Xpath的條件之一。

\indent
(少一行)

\section{研究目標}
\indent
在呂昭陞論文中主要提出在網頁自動化測試下如何改善Xpath表達式,
其中有詳細解釋說,相較於用階層的表示元件位置,若每一層都用相異或較為獨特的條件去限制,Xpath能夠成功定位的成功率較高。
以相同的規範的前提下,使用者需要先找到適合的條件限制,才能寫出一個穩定性高的表達式,
但若在一個HTML大的HTMLL架構中,如何挑選在和元件有互動後找到好的條件來放入表達式中會是一個比較花時間和經驗的事。
因此本論文會針對這方面的困難提供一個瀏覽器擴充工具,找出符合當下情況中和元件互動前後找出兩個HTML文本不同的部分,並且標示出來讓使用者做條件的選擇,
會使得使用者可以過濾掉很多不需要參考的元件,把可使用的資訊量縮小。

\section{論文組織架構}
\indent
%可能全部寫完需待修正
本論文一共有X章節,
其章節構成如下:
第二章介紹相關背景知識、技術及使用工具,
第三章會探討如何實作XXX,
第四章XXX,
第五章XXX,
最後第五章會做總結,討論使用此方法後的結果與未來展望。

% 其中介紹了欲擴充之重構方法及其流程,並且介紹如何將重構方法
% 擴充至RF Refactoring中,
% 第四章將會以第三章所設計之方法進行實作,其中包含了重構流程及外掛程式之實作,
% 第五章將以實際Robot Framework測試腳本介紹重構案例,
% 並且邀請測試團隊成員分別使用Visual Studio Code搜尋取代工具及本論文擴充後之重構工具進行重構,後續比較使用之差異。
% 第五章總結,討論應用此方法後的結果與未來展望。