\chapter{緒論}
\section{研究背景與動機}
\indent
在建構一個Web Application時常會用HTML技術建構 Web Application的使用者介面,通常以網站居多。
網站通常是由HTML、CSS和JavaScript來組成的,
其中HTML是網站的主要架構,CSS可以調整網站視覺化的呈現,JavaScript可以使物件有可互動的特性。
在打開一個新網站後,該頁面的HTML文本會因為使用者的一些操作行為,像是點擊元件、鼠標懸浮在元件上...等等互動行為,
讓HTML中裡面的元件做增加、刪除和修改的狀態,使得整個HTML相較互動前會有一些變化。
例如:在點擊網站中的按鈕,會跳出"你已點擊"一行字,
這代表在你點擊後,在JavaScript程式中觸發了按鈕"click"的動作,使得程式會幫你在HTML文本中添加了一個"你已點擊"的文字元件。
而Xpath是一個基於XML樹狀結構,經由使用者設定的條件及規則,找到想獲得的一個或多個元件,
表達式不同的寫法或不同的限制條件會使得兩個表達式是不同,但找到的元件是相同的。

\indent
一般在開發完前端程式後,需要檢測程式是否有達成預期效果,
為了減少人工成本,通常測試的部分都是以測試腳本模擬人的行為來進行不斷的檢測,來判斷目前版本程式是否有正常。
網頁自動化測試腳本主要是模擬使用者操作網站,利用程式的方式,檢測網站前後端是否有出現一些錯誤,從而來進行一些修改。
在撰寫一個網頁自動化測試腳本,和網頁中的元件互動是必須的,
互動完通常我們常常需要定位一個或多個元件去等它出現或消失,透過這個規律來設計出一連串的操作。
為了要程式知道設計者想要查看哪個元件當下的狀態,會使用Xpath表達式來找出元件位置,
但通常如果Xpath的條件設定的不好,常常會因為網頁結構改變而需要去多次調整Xpath來避免錯誤。

\indent
通常我們為了要增加Xpath的穩定性,我們會尋找會因為某個使用者互動而影響元件屬性或一些固有的屬性來在Xpath裡面去限制它的條件,
而如果HTML文本較大,我們在尋找時範圍會變大,花的時間成本也相對高,
因此為了要改善這方面的問題,需要一個能夠快速過濾一些不相關且沒有變化的元件,
讓我們方便挑選出有產生改變的元件來當我們的Xpath的條件之一。

\section{研究目標}
\indent
在呂昭陞論文中主要提出在網頁自動化測試下如何改善Xpath表達式,
其中有詳細解釋說,相較於用階層的表示元件位置,若每一層都用相異或較為獨特的條件去限制,Xpath能夠成功定位的成功率較高。
以相同的規範的前提下,使用者需要先找到適合的條件限制,才能寫出一個穩定性高的表達式,
若在一個架構稍微繁瑣的HTMLL中,如何挑選在和元件有互動後找到好的條件來放入表達式中會是一個需要有足夠經驗以及充足時間來尋找的事。
因此本論文會針對此方面的困難提供一個瀏覽器擴充工具,
找出符合當下情況中和元件互動前後找出兩個HTML文本不同的部分,利用Timer和Filter...等等的小工具並且標示出來讓使用者做條件的選擇,
會使得使用者可以過濾掉很多不需要參考的元件,把可使用的資訊量縮小。

\section{論文組織架構}
\indent
%可能全部寫完需待修正
本論文一共有X章節,
其章節構成如下:
第二章介紹相關背景知識、技術及使用工具,
第三章會探討如何實作XXX,
第四章XXX,
第五章XXX,
最後第五章會做總結,討論使用此方法後的結果與未來展望。

% 其中介紹了欲擴充之重構方法及其流程,並且介紹如何將重構方法
% 擴充至RF Refactoring中,
% 第四章將會以第三章所設計之方法進行實作,其中包含了重構流程及外掛程式之實作,
% 第五章將以實際Robot Framework測試腳本介紹重構案例,
% 並且邀請測試團隊成員分別使用Visual Studio Code搜尋取代工具及本論文擴充後之重構工具進行重構,後續比較使用之差異。
% 第五章總結,討論應用此方法後的結果與未來展望。