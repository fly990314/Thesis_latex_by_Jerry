\chapter{緒論}
\section{研究背景與動機}\label{s1.1}
\indent
建構一個Web Application時常會使用HTML技術建構一個Web Application的使用者介面\cite{Web-application},大部分都以網站居多。
以網站舉例,通常是由HTML、CSS和JavaScript來組成的\cite{HTML-CSS-JAVASCRIPT},
其中HTML是網站的主要架構,而CSS可以調整網站視覺化的呈現,JavaScript則是能讓物件擁有可互動的特性。

在一個已被開啟的網頁中,若使用者操作該網頁並讓畫面產生相對應的變化,
代表這些因操作而產生的變化有可能會使HTML內容與操作前產生些許不同。
例如:點擊網站中的按鈕,會跳出``你已點擊''一行字,
這代表點擊後,在JavaScript程式中觸發了``按鈕Click''的事件,並在HTML中添加``你已點擊''的文字元件;
又或者是讓鼠標懸浮在元件上後會顯示下拉式選單,
也代表著鼠標停滯在元件上方時觸發了``元件Hover''的事件,使得下拉選單自動在畫面中展開。
而XPath是一個基於XML樹狀結構,可以經由使用者設定的條件及規則,定位一個或多個元件,
表達式中不同的路徑選擇或不同的限制條件會使得兩個表達式不同,但可能找到的元件是相同的。

\indent
一般撰寫網頁完成後,需要檢查網頁中的互動功能是否正常,手動測試會是最直接且簡單的作法,
但隨著網頁持續更新,若程式添加了新功能卻影響到了舊的互動行為且沒被發現,可能會使該網站越來越多錯誤。
為了降低人工成本以及避免既有的功能出錯,可以利用網頁自動化測試腳本模擬使用者操作網站,
並定期的執行測試程式來反覆檢測網站前後端是否有出現錯誤。
在撰寫網頁自動化測試腳本時,頻繁地和網頁元件互動是無可避免的。
為了確保腳本中每次互動都是成功的,
可以藉由抓取網頁中因為互動而產生變化的元件,檢查元件互動後的狀態來判斷網頁功能是否正常。

\indent
為確定元件成功產生互動,測試人員大多會將因互動而產生變化的元件屬性設定為XPath表達式的路徑條件



元件互動後的狀態可以利用XPath路徑表達式中的條件限制來判斷,
通常測試人員在設計條件時,會使用因為互動而產生變化的元件屬性,
若網頁功能故障,
應當變化的元件屬性因為沒有互動使得不會有任何變化,
測試腳本也會因為XPath路徑條件未滿足而導致程式失敗。
由此可知,在設計XPath路徑條件時,常常會因為要去找出能判斷互動行為的元件屬性,而使用了很多的時間在尋找適合的條件。
因此,如果有工具可以協助測試人員有效率地找出適合判斷互動行為是否正常的元件屬性,能讓撰寫測試腳本的成本大幅降低。

\section{研究目標}\label{s1.2}
\indent
在呂昭陞論文\cite{LIU-Thesis}中主要提出在網頁自動化測試下如何改善XPath表達式,
其中有詳細解釋用元件層數來找出元件位置的方法,穩定度會比每一層都用相異或較為獨特的條件去限制路徑的方法低。
因此,在相同規範的前提下,使用者需要先找到適合的條件限制,才能寫出一個穩定性高的表達式。

在一個架構繁瑣的HTML中,
若要設計一個與元件互動後可以判斷元件有確實動作的XPath表達式,
路徑條件挑選會是一個需要有足夠經驗以及充足時間來尋找的事。
因此本論文會針對此方面的困難建構瀏覽器擴充工具,
藉由HTML比對,找出互動前後HTML的相異之處,
並提供Timer和Filter等等的功能,
使它操作更為便利、更精準的篩選結果,
最後標記出有變化的元件屬性,讓使用者挑選適合當下XPath表達式的路徑條件。

\section{論文組織架構}\label{s1.3}
\indent
本論文一共有五個章節,
其章節構成如下:
第二章將介紹相關背景知識、技術及使用工具,
第三章介紹如何實做此HTML比對擴充工具,
其中包括HTML比對之方法以及將HTML比對融入瀏覽器擴充元件之實作,

第四章將以範例網站來當作範例,
邀請測試團隊的成員分別以只使用Chrome Developer Tools以及Chrome Developer Tools加上本論文HTML比對之瀏覽器擴充元件的兩種操作方式進行XPath表達式條件挑選,
後續比較使用前後之差異,
最後第五章做出總結,討論使用此方法所得到的結果與未來展望。